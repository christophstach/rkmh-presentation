\section{Einleitung}

\begin{frame}{Einleitung}
    \begin{itemize}
        \item DNA: (A)denin, (T)hymin, (G)uanin und (C)ytosin
        \item Sequencing-Reads: Länge von \textasciitilde 250 Proteien-Basenpaaren
        \item Datensätze: FASTQ, FASTA
    \end{itemize}
\end{frame}


\begin{frame}[fragile]{Einleitung - FASTA Beispiel\footnote{\url{http://www.cbs.dtu.dk/services/NetGene2/fasta.php}}}
    \begin{minted}{text}
>HSBGPG Human gene for bone gla protein (BGP)
GGCAGATTCCCCCTAGACCCGCCCGCACCATGGTCAGGCATGCCCCTCCTCATCGCTGGGCACAGCCCAGAGGGT
ATAAACAGTGCTGGAGGCTGGCGGGGCAGGCCAGCTGAGTCCTGAGCAGCAGCCCAGCGCAGCCACCGAGACACC
ATGAGAGCCCTCACACTCCTCGCCCTATTGGCCCTGGCCGCACTTTGCATCGCTGGCCAGGCAGGTGAGTGCCCC
CACCTCCCCTCAGGCCGCATTGCAGTGGGGGCTGAGAGGAGGAAGCACCATGGCCCACCTCTTCTCACCCCTTTG
GCTGGCAGTCCCTTTGCAGTCTAACCACCTTGTTGCAGGCTCAATCCATTTGCCCCAGCTCTGCCCTTGCAGAGG
GAGAGGAGGGAAGAGCAAGCTGCCCGAGACGCAGGGGAAGGAGGATGAGGGCCCTGGGGATGAGCTGGGGTGAAC
CAGGCTCCCTTTCCTTTGCAGGTGCGAAGCCCAGCGGTGCAGAGTCCAGCAAAGGTGCAGGTATGAGGATGGACC
TGATGGGTTCCTGGACCCTCCCCTCTCACCCTGGTCCCTCAGTCTCATTCCCCCACTCCTGCCACCTCCTGTCTG
GCCATCAGGAAGGCCAGCCTGCTCCCCACCTGATCCTCCCAAACCCAGAGCCACCTGATGCCTGCCCCTCTGCTC
CACAGCCTTTGTGTCCAAGCAGGAGGGCAGCGAGGTAGTGAAGAGACCCAGGCGCTACCTGTATCAATGGCTGGG
GTGAGAGAAAAGGCAGAGCTGGGCCAAGGCCCTGCCTCTCCGGGATGGTCTGTGGGGGAGCTGCAGCAGGGAGTG
GCCTCTCTGGGTTGTGGTGGGGGTACAGGCAGCCTGCCCTGGTGGGCACCCTGGAGCCCCATGTGTAGGGAGAGG
AGGGATGGGCATTTTGCACGGGGGCTGATGCCACCACGTCGGGTGTCTCAGAGCCCCAGTCCCCTACCCGGATCC
CCTGGAGCCCAGGAGGGAGGTGTGTGAGCTCAATCCGGACTGTGACGAGTTGGCTGACCACATCGGCTTTCAGGA
GGCCTATCGGCGCTTCTACGGCCCGGTCTAGGGTGTCGCTCTGCTGGCCTGGCCGGCAACCCCAGTTCTGCTCCT
CTCCAGGCACCCTTCTTTCCTCTTCCCCTTGCCCTTGCCCTGACCTCCCAGCCCTATGGATGTGGGGTCCCCATC
ATCCCAGCTGCTCCCAAATAAACTCCAGAAG
    \end{minted}
\end{frame}

\begin{frame}[fragile]{Einleitung - FASTQ Beispiel\footnote{\url{https://github.com/edawson/rkmh_sim_data}}}
    \begin{minted}{text}
631232382/31100220/2202403222222222223125251225244225125230222124223302
@HPV16|Z109|B2_601_1122_0_1_0_0_15:1:0_11:0:0_71933/2
CCCCCACTTCCACCACTTATACTGCTACATGGTGTTTCAGTCTCATGGCGCCCTTCTACCTGTAACGATTC
+
236520202/2132221222221541022222022222323023213421222445242222/12250234
@HPV16|Z109|B2_388_953_0_1_0_0_10:0:0_11:0:0_e0828/2
ATCTACGGACTAATATTATTGTCTACACATCCACTAATATCACTCAAGTGGACTACCCAAATACTTTCGTT
+
3232112242221235221/3321543020232443253222/3122203042202032425044/07245
@HPV16|Z109|B2_4780_4199_1_0_0_0_11:0:0_3:1:0_ba60c/2
TTTATTGTTTGTTTTGTTTGTTTTTTAAATAAACTGTTATTACTTAACAATGCGACACAAACGTTCTGCAA
+
221122234222229205202222232221262370322524352255/3212123232303234422330
@HPV16|Z109|B2_5953_6522_0_1_0_0_13:0:0_11:1:0_cc11b/1
GTGTAGGTGTTGAGGTAGGTCGCGGTCAGCCATTAGGTGTGGGCATTAGTGGCCATCCTTTATTAAATAAA
    \end{minted}
\end{frame}

\section{Rkmh}

\begin{frame}{Rkmh I}
    \begin{itemize}
        \item \textbf{Viral coinfection analysis using a MinHash toolkit \cite{rkmh}}
        \item Human papillomavirus (HPV)
        \item HPV hat viele unterschiedliche Typen bzw. Abstammungslinien
        \item Co-Infektionen verschiedener Abstammungslinien sind üblich
        \item Davon sind nur einige hoch krebserzeugend
    \end{itemize}
\end{frame}

\begin{frame}{Rkmh II}
    \begin{itemize}
        \item rkmh: klassifiziert Reads aus FASTQ-Dateien
        \item Jeder Read wird mit einer Menge von Referenzgenomen verglichen
        \item rkmh weist jedem Read das Referenzgenom zu mit dem es die höchste Übereinstimmung hat
        \item Für den Vergleich bedient sich rkmh der MinHash-Methode  aus dem Bereich der Textanalyse
        \item \textbf{On the Resemblance and Containment of Documents \cite{minhash}}
    \end{itemize}
\end{frame}