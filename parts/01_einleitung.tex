\section{Einleitung}

\begin{frame}
    \frametitle{Einleitung 1}

    

    \begin{itemize}
        \item DNA: (A)denin, (T)hymin, (G)uanin und (C)ytosin
        \item Sequencing-Reads: Länge von 
        \item Datensätze: FASTQ, FASTA
    \end{itemize}

    
\end{frame}


\begin{frame}
    \frametitle{Einleitung 2}

    \begin{itemize}
        \item Viral coinfection analysis using a MinHash toolkit
        \item Human papillomavirus (HPV)
        \item HPV hat viele unterschiedliche Typen bzw. Abstammungslinien
        \item Davon sind nur einige hoch krebserzeugend
        \item Co-Infektionen verschiedener Abstammungslinien sind üblich
    \end{itemize}
\end{frame}

\begin{frame}
    \frametitle{Einleitung 3}

    
    \begin{itemize}
        \item rkmh: klassifiziert reads aus FASTQ-Dateien
        \item Jeder read wird mit einer Menge von Referenzgenomen verglichen
        \item rkmh weist jedem read das Referenzgenom zu mit dem es die höchste Übereinstimmung hat
        \item Für den Vergleich bedient sich rkmh der MinHash-Methode aus dem Bereich der Textanalyse
    \end{itemize}
\end{frame}

\begin{frame}
    \frametitle{Einleitung 4}

    
    \begin{tikzpicture}

        \def \n {1,2,3}
        \def \radius {3cm}
        \def \margin {8} % margin in angles, depends on the radius
        
        \foreach \c [count=\x from 0] in {{1,f},{2,o},{3,o},{4,b},{5,a},{6,r}} 
        {
            \node[draw, circle] at ({360/\n * (\x - 1)}:\radius) {$\c$};
            \draw[->, >=latex] ({360/\n * (\x - 1)+\margin}:\radius) 
              arc ({360/\n * (\x - 1)+\margin}:{360/\n * (\x)-\margin}:\radius);
        }
    \end{tikzpicture}
\end{frame}