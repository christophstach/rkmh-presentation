\section{MinHash}

\begin{frame}{MinHash I}
    \begin{itemize}
        \item Algorithmus aus der Textanalyse
        \item Wurde erstmalig verwendet um eine große Menge von Webseiten miteinader zu vergleichen
        \item Arbeitet mit k-shingles
        \item Nährt den Jaccard-Koeffizienten
    \end{itemize}
\end{frame}

\begin{frame}{MinHash II}
    Die Mengen der geshingelten werden als Matrix $ M $ dargestellt.
    
    \begin{example}
        \begin{equation*}
            M = 
            \kbordermatrix{
                & D_1 & D_2 \\
                ATG & 1 & 0 \\
                TGG & 1 & 1 \\
                GGC & 1 & 1 \\
                GCA & 1 & 1 \\
                CAT & 1 & 0 \\
                CAG & 0 & 1 \\
                AGT & 0 & 1
            }
        \end{equation*}
    \end{example}
\end{frame}

\begin{frame}{MinHash III}
    Matrix $ P $ aus zufälligen Zeilenindexpermutationen bilden.
    
    \begin{example}
        \begin{equation*}
            P =
            \kbordermatrix{
                & P_1 & P_2 & P_3 & P_4 & P_5 \\
                & 0 & 3 & 5 & 4 & 4 \\
                & 1 & 6 & 4 & 1 & 0 \\
                & 2 & 2 & 1 & 2 & 3 \\
                & 3 & 5 & 3 & 6 & 1 \\
                & 4 & 1 & 2 & 3 & 6 \\
                & 5 & 0 & 0 & 5 & 2 \\
                & 6 & 4 & 6 & 0 & 5
            }
        \end{equation*}
    \end{example}
\end{frame}

\begin{frame}{MinHash IV}
    Jedes $ M $ wird nach der Zeilenindexpermutation sortiert. Der Index der ersten Zeile in $ M $ welcher den Wert $ 1 $ hat wird in die Signatur $ S $ übertragen.


    \begin{example}
        \begin{equation*}
            S = 
            \kbordermatrix{
                & D_1 & D_2 \\
                P_1 & 0 & 1 \\
                P_2 & 1 & 0 \\
                P_3 & 1 & 1 \\
                P_4 & 1 & 1 \\
                P_5 & 1 & 1
            }
        \end{equation*}
    \end{example}
\end{frame}

\begin{frame}{MinHash V}
    \begin{itemize}
        \item Problem: Vorhalten der Permutationen verbraucht viel Arbeitsspeicher
        \item Lösung: Benutzung von Hash-Funktionen
        \item Mit Hash-Funktionen können Daten einer Menge \textit{deterministisch-zufällig} sortiert werden
        \item Beispiel: Hash-Funktion $ h(x) = (2x+1) \mod 6 $ und Menge $ S = \{ 10,11,30,41 \} $
        \item Pro Permutation andere Hash-Funktion
    \end{itemize}
\end{frame}

\begin{frame}{MinHash VI}
    \begin{example}
        \begin{equation*}
            \begin{split}
                h(10) = (2 * 10 + 1) \mod 6 = 3 \\
                h(11) = (2 * 11 + 1) \mod 6 = 5 \\
                h(30) = (2 * 30 + 1) \mod 6 = 1 \\
                h(41) = (2 * 41 + 1) \mod 6 = 5
            \end{split}
        \end{equation*}
    \end{example}
    
        \begin{table}[H]
            \centering
            \begin{tabular}{| c | c | c | c | c | c | c |}
                \hline
                \textbf{Hash} & \textbf{1} & \textbf{2} & \textbf{3} & \textbf{4} & \textbf{5} & \textbf{6}  \\
                \hline
                \multirow{2}{*}{\textbf{Elemente}}   & 30 &    & 10 &    & 11 & \\
                &    &    &    &    & 41  &\\
                \hline
            \end{tabular}    
            \caption{Tashtabelle der Menge $ S $. Jedes Element von $ S $ hat eine unterschiedlichen Hash. Einige Werte erzeugen Kollisionen.}
        \end{table} 
\end{frame}