\section{Shingling}

\begin{frame}{Shingling I}
    \begin{itemize}
        \item n-grams, k-mers, k-shingles
        \item Sequenzen aus Wörtern oder Buchstaben werden geshingelt
        \item Durch Shingling entsteht eine Menge von Shinglen ohne doppelte Vorkommen
        \item Geshingelte Sequenzen können besser mit einander verglichen werden
    \end{itemize}
\end{frame}


\begin{frame}{Shingling - Beispiel}
    Gegeben sind zwei Dokumente $ A = ``ATGGTAT" $ und $ B = ``TGGCAGT" $ und eine Shinglelänge von $ k = 3 $.
    
    \begin{example}
        \begin{equation*}
            \begin{split}
                S_3(A) = \{ATG, TGG, GGC, GCA, CAT\} \\
                S_3(B) = \{TGG, GGC, GdA, CAG, AGT\} \\
                S_3(A) \cup S_3(B) = \{ATG, TGG, GGC, GCA, CAT, CAG, AGT\}
            \end{split}
        \end{equation*}
    \end{example}
\end{frame}